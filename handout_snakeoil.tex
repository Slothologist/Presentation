\documentclass[
	11pt,
	DIV=15,
	parskip=full,    % spacing value / method for paragraphs
]{scrartcl}

\usepackage[german]{babel}  
\usepackage[utf8]{inputenc}	%dt Sonderzeichen wie ß
\usepackage{enumerate}
\usepackage{setspace}
\usepackage{listings}
\usepackage{todonotes}
\usepackage{graphicx}
\usepackage{titlesec}
\usepackage{lmodern}
\usepackage{caption}
\usepackage{color}

%\titlespacing{\subsection}{12pt}{-0.5em}{-1.5em}
%\RedeclareSectionCommand[beforeskip=-.75\baselineskip,afterskip=.5\baselineskip]{subsection}


\begin{document}
	\begin{center}
		\begin{onehalfspace}
			\Large{Schlangenöl\\Viren und wie man sie nicht vermeidet}\\
			\large{Digitale Kommunikation und Internetdienste II}\\
		\end{onehalfspace}
		\large{Robert Feldhans (rfeldhans@techfak.uni-bielefeld.de)}\\
		\large{Dominik Sixt (dsixt@techfak.uni-bielefeld.de)}
		\end{center}
	\hrulefill
	
	\subsection*{Schlangenöl}
	\begin{itemize}
		\itemsep-1pt
		\item Beginn: Heilmittel für Gebrechen aller Art, von Wunderheilern im Wilden Westen eingesetzt
		\item Heute: versprochenes Wundermittel, welches oft dem technischen Bereich entstammt (z.B. Kryptografie und Antiviren-Software)
	\end{itemize}
	
	\subsection*{Malware}
	Oberbegriff für Software, welche schädliche Funktionen ausführen
	
	\begin{itemize}
		\itemsep-1pt
		\item Viren: Schadprogramm, welches sich verbreitet, in dem es sich in andere Software einschleust. Durch das Kopieren dieser wird der Virus passiv verbreitet (und dabei oft nur lokal)
		\item Würmer: Schadprogramm, welches sich aktiv ausbreitet, in dem es Sicherheitsprobleme ausnutzt. Für Nutzer kaum unterschiedlich zu Viren
		\item Trojanische Pferde: Schadprogramm, welches sich als nützliche Anwendung tarnt und ohne Wisse des Anwenders (auch) schädliche Funktionen ausführt
		\item Ransomware: Trojaner, welcher Zugriffe auf persönliche Daten oder derer des Betriebssystems verhindert. Für eine Entschlüsselung dieser wird schließlich Lösegeld gefordert
		\item ...
		\item nicht: fehlerhafte Software	
	\end{itemize}
	
	\subsection*{Hauptverteilwege von Malware}
	\begin{itemize}
		\itemsep-1pt
		\item E-Mail-Dateianhänge
		\item Drive-by-Downloads
		\item Datenträger
		\item Netzwerke
	\end{itemize}
	
	$\Rightarrow$ \textcolor{red}{Vermeintliche Lösung:} Antiviren-Programm installieren!
	
	\subsection*{Verbreitung von AV-Programmen}
	\begin{itemize}
		\itemsep-1pt
		%\item{AVG Antivirus App 4.6 Millionen Downloads (Android)}
		%\item{360 Security 12.5 Millionen Downloads (Android)}
		\item{AVG Antivirus Free 22.3M}
		\item{avast Free Antivirus 19.4M}
		\item{Ad-Aware Free Antivirus 9.2M}
	\end{itemize}
	jeweils in Millionen Downloads bei Chip (Windows)
	
	\subsection*{Hauptmaßnahmen der AV-Programme}
	\begin{itemize}
		\itemsep-1pt
		\item Virtualisierung
		\item Firewall
		\item AV-Browser
	\end{itemize}
	
	$\rightarrow$ \textcolor{red}{Diverse Fehlschläge} in der Implementierung von AV-Programmen, welche die Sicherheitslücken (teilweise) hinzufügen. Siehe hierfür die Präsentation
	
	\subsection*{Wirksame Maßnahmen gegen Viren}
	\begin{itemize}
		\itemsep-1pt
		\item Präventives Verhalten erlernen
		\item Software mit großem Sicherheitsrisiko meiden
		\item Fremdcode von Webseiten minimieren
		\item Richtige Wahl/Konfiguration des Betriebssystems
	\end{itemize}
	
	
	
	
	
	\titlespacing{\subsection}{12pt}{-0.5em}{-0.5em}
	
	\subsection*{Quellen}
	\begin{itemize}
		\itemsep-7pt
		\item blog.fefe.de
		\item itwissen.info/definition/lexikon/ (diverse Definitionen)
		\item de.wikipedia.org/wiki/Schadprogramm
		\item https://de.wikipedia.org/wiki/Antivirenprogramm
		\item www.chip.de (diverse AV-Downloadseiten) 
		\item https://de.wikipedia.org/wiki/Firewall
	\end{itemize}
	

	
\end{document}
